\documentclass[]{article}

\usepackage[utf8]{inputenc}
\usepackage{fancyhdr,verbatim,parskip}
\usepackage[margin=1in]{geometry}
\usepackage{enumitem}
\usepackage{pdfpages}
\usepackage{graphicx}
\usepackage{mathtools,amsmath,amssymb,amsfonts} % for math stuff

% for code
\usepackage{minted}

\pagestyle{fancy}
\fancyhf{}
\rhead{YOUR NAME, STUDENT \#}
\lhead{COURSE NAME TODO}
\rfoot{Page \thepage}

% Optional: add word count info
% Input the word count
% \immediate\write18{detex \jobname | wc -w > \jobname_wordcount.txt}
% \lfoot{{\input{\jobname_wordcount.txt}} words (using \texttt{detex})}

% Potentially add a bibliography
%\usepackage{biblatex}
%\addbibresource{refs.bib}

\begin{document}

% Potentially include a cover sheet
% \includepdf[pages={1}]{coversheet.pdf}
	
\begin{comment}
I like to start with a block comment outline of what I will write about
\end{comment}
	
	
\section*{Title}
Some content, more content.

Some code:
\begin{minted}[tabsize=4,python3,breaklines]{python}	
	import math
	print(math.sqrt(2.0))
\end{minted}

Refer to a figure (Figure~\ref{fig:1}).
\begin{figure}[h]
		\centering
		\includegraphics[width=0.6\linewidth]{example-image-a}
		\caption{a caption}
		\label{fig:1}
\end{figure}

% Answer questions with letter numbers
\begin{enumerate}[label=(\alph*)]
	% =================================================================
	% Part a
	% =================================================================
	\item\label{part-a}
	
	Part~\ref{part-a} refers to Figure~\ref{fig:1}.
	
	
	
	
	
	
	% =================================================================
	% Part b
	% =================================================================
	\newpage
	\item\label{part-b}
	Another answer.
\end{enumerate}


%\printbibliography

\end{document}
